\documentclass[times, utf8, diplomski]{fer}
\usepackage{booktabs}

\begin{document}


\title{ Raspoznavanje aktivnosti osoba na temelju siluete }


\author{ \begin{tabular}{ l }
	Kuman Stipe \\
	Meštrović Stjepan \\
	Mujić Azzaro \\
	Mužar Irena \\
	Novković Igor \\
	Skukan Luka \\
	Venanzoni Andrea \\
\end{tabular}  }

\maketitle

% Dodavanje zahvale ili prazne stranice. Ako ne želite dodati zahvalu, naredbu ostavite radi prazne stranice.
\zahvala{}

\tableofcontents


\chapter{Projektni zadatak}

(do 10 stranica)

\section{Opis projektnog zadatka}

Opis problema koji zadatak obuhvaća. Što su ulazni podaci, a što zahtjevani izlaz? Koncepti/algoritmi koji se obavezno moraju upotrijebiti?

\section{Pregled i opis srodnih rješenja}

Iscrpan pregled srodne literature s predloženim rješenjima. Opis postojećih ispitnih baza (linkovi na javno dostupne baze).

\section{Konceptualno rješenje zadatka}

Opisati niz algoritama i koncepata koji će se koristiti u rješavanju i to redom kojim će se koristiti. Nabrojiti ulazne podatke u niz i sve međurezultate do rješenja.


\chapter{Postupak rješavanja zadatka}

(do 10 stranica)

Navesti numerirani slijed koraka rješavanja. Npr.: 1. Dobivanje binarne slike iz slike u boji, 2. Segmentacija objekata na slici, 3. Nalaženje rubova u slici ...

\section{Naziv prvog koraka}

Za svaki korak napisati što su ulazi i što su izlazi. Popisati sve algoritme/ koncepte koji se u tom koraku koriste za pretvorbu ulaza u izlaz. Navesti sve probleme koji su se pojavili u pojedinom koraku i kako su riješeni. Pojedinačno opisati svaki korišteni algoritam/koncept:

\subsection{Naziv prvog algoritma}

Opis/koraci/matematička formulacija, prednosti i mane, ulazi i izlazi te korišteni parametri.

\subsection{Naziv drugog algoritma}

Opis/koraci/matematička formulacija, prednosti i mane, ulazi i izlazi te korišteni parametri.

\section{Naziv drugog koraka}

\chapter{Ispitivanje rješenja}

(do 10 stranica)

\section{Ispitna baza}

Opisati ispitnu bazu, tipove i broj različitih uzoraka u bazi te na koji su način uzorci iz baze korišteni prilikom učenja i ispitivanja rješenja projektnog zadatka. 

\section{Rezultati učenja i ispitivanja}

Prikazati statističke podatke o uspješnosti rješenja prilikom učenja/ispitivanja te opisati eksperimente na temelju kojih su podaci dobiveni.

\section{Analiza rezultata}

Analizirati uzroke rezultata ispitivanja, povezati sa uzorcima u bazi i algoritmima korištenim u rješenju. Raspraviti moguća poboljšanja.

\chapter{Opis programske implementacije rješenja}

(do 5 stranice)

Opisati sučelje programske implementacije i način korištenja implementacije.


\chapter{Zaključak}

(do 2 stranice)

Ocijeniti uspješnost implementacije, navesti budući rad u smislu potrebnih poboljšanja. 


\chapter{Literatura}

1. Ime i prezime autora: Naziv časopisa vol. br. godina izdanja, pp od-do (npr. pp 486-492)/knjige/članka/web resursa (s linkom i datumom pristupa web resursu)
...
.
.
DVD/CD  
.
kompletan tekst projekta
izvorni kod projekta
exe verzija
readme file – upute za korištenje i pokretanje programa
.
baze slika (sve koje su korištene)
E-oblik članaka koji su korišteni za izradu projekta
primjeri obrade
..





%%\bibliography{literatura}
%%\bibliographystyle{fer}

\begin{sazetak}
Sažetak na hrvatskom jeziku.

\kljucnerijeci{Ključne riječi, odvojene zarezima.}
\end{sazetak}

% TODO: Navedite naslov na engleskom jeziku.
\engtitle{Title}
\begin{abstract}
Abstract.

\keywords{Keywords.}
\end{abstract}

\end{document}
